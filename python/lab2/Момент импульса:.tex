Момент импульса:
\[ M = r \times p \]

Рассмотрим систему из двух частиц

F - внешняя сила (сумма внешних сил)

\[ m_1 \cdot \dot  v_1 = F_{12} + F_1 \]

\[ m_2 \cdot \dot  v_2 = F_{21} + F_2 \]

Векторно умножим (слева) на \( \vec r_1, \vec r_2\) :

\[ m_1 \cdot  r_1 \times \dot v_1 = r_1 \times F_{12} + r_1 \times F_1 \]

\[ m_2 \cdot   r_2 \times \dot v_2 = r_2 \times F_{21} + r_2 \times F_2 \]

\( r \times \dot v = \frac{d}{dt} r \times v \; ; \; m \cdot v = p \)

\[   \frac{d}{dt} \cdot  r_1  \times ( m_1 \cdot v_1) = r_1 \times F_{12} + r_1 \times F_1 \]

\[ m_2 \cdot \frac{d}{dt} \cdot   r_2 \times (m_2 \cdot v_2) = - r_2 \times F_{12} + r_2 \times F_2 \]

Сложим вместе два уравнения:

\[ \frac{d}{dt} \left\{ r_1 \timex p_1 + r_2 \times p_2 \right\} = r_1 \times F_1 + r_2 \times F_2 \]

Система замкнута, правая часть равна нулю

\[ r_1 \times p_1 + r_2 \times p_2 = const  \]


Момент импульса:
\[ M = r \times p \]


------------------------------------------


\[ N = -r_1 \times F_2 + r_2 \times F_2 = (r_2 - r_1) \times F_2 = r_{12} \times F_2 \]


Это выражение не зависит от точки.
Вектор момента пары сил перпендикулярен плоскости, в которой лежат силы.

Моменты внутренних сил уравновешивают друг друга

\[ \sum N_{внутр} = 0 \]


\[ \frac{d}{dt} M = \sum N^* \]

\( N^* \) со звездочкой, как всегда, внешние

При отсутсвии внешних M неизменяется.


----------------------------------------

Движение в центральном полу сил
